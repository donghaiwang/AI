\documentclass[12pt]{book}  
\usepackage{CJKutf8}%中文支持  
\usepackage{indentfirst}%设置段落缩进.如果没有,section后面正文则顶格显示.不缩进.  
\linespread{1.5}%以將行距加大為1.5倍:  
\usepackage{amsmath}  
\usepackage{amssymb}  
\title{地名的由来}%自行添加    
\author{王海东}  
\date{2018.6}  
  

\begin{document}  
\begin{CJK*}{UTF8}{gbsn}%中文支持  
\maketitle  
\thispagestyle{empty} % 当前页不显示页码  
\clearpage
  
%\pagestyle{headings}  
\setcounter{page}{1}  
\pagenumbering{Roman}  
  
  
\clearpage  
\tableofcontents  
\newpage  

\setcounter{page}{1}
\pagenumbering{arabic}

\chapter{安徽}
\section{蚌埠市}
意为“盛产\textbf{蚌}珠的港\textbf{埠}”,蚌埠由此别称为珠城。

\chapter{湖南}

\section{常德}
\subsection{安乡县}
明隆庆《岳州府志》载,县境“左挹洞庭,右接兰澧”,取“洞庭兰澧诸水各安其流”之义名县。

\subsection{鼎城区}
清同治《武陵县志》载:宋乾德二年(964年),军降团练,大中祥符五年(1012年),州(本唐、五代朗州)改鼎州。
“鼎州”名称的由来,据说是因武陵县境,在沅、澧二水汇合处有鼎水,“昔有神鼎出乎其间”,故以名州。

\subsection{汉寿县}
东汉阳嘉三年(134年)改索县为汉寿县,取“汉朝万寿无疆”之意,三国时被改为“吴寿县”,后多有更改,1912年民国建立后以“驱除靼虏,汉室复兴”之意复名汉寿至今。


\clearpage  %要生成目录,需要添加这个\newpage  
\end{CJK*}%中文支持  
\end{document} 